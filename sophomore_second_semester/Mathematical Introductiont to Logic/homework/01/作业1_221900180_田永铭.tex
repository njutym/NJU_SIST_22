% 作业模板
\documentclass[UTF8,10pt,a4paper]{article}
\usepackage{ctex}
\usepackage{graphicx}
% \catcode`\。=\active
% \newcommand{。}{.}
\newcommand{\CourseName}{数理逻辑}
\newcommand{\Semester}{2023-2024学年第二学期}
\newcommand{\ProjectName}{作业 1}
\newcommand{\DueTimeType}{完成日期}
\newcommand{\DueTime}{\today}
\newcommand{\StudentName}{田永铭}
\newcommand{\StudentID}{221900180}
\usepackage[vmargin=1in,hmargin=.5in]{geometry}
\usepackage{fancyhdr}
\usepackage{lastpage}
\usepackage{calc}
\pagestyle{fancy}
\fancyhf{}
\fancyhead[L]{\CourseName}
\fancyhead[C]{\ProjectName}
\fancyhead[R]{\StudentName}
\fancyfoot[R]{\thepage\ / \pageref{LastPage}}
\setlength\headheight{12pt}
\fancypagestyle{FirstPageStyle}{
	\fancyhf{}
	\fancyhead[L]{\CourseName\\
		\Semester}
	\fancyhead[C]{{\Huge\bfseries\ProjectName}\\
		\DueTimeType\ : \DueTime}
	\fancyhead[R]{姓名 : \makebox[\widthof{\StudentID}][s]{\StudentName}\\
		学号 : \StudentID\\
		成绩 :\makebox[\widthof{\StudentID}]{}}
	\fancyfoot[R]{\thepage\ / \pageref{LastPage}}
	\setlength\headheight{36pt}
}
\usepackage{amsmath,amssymb,amsthm,bm}
\allowdisplaybreaks[4]
\newtheoremstyle{Problem}
{}
{}
{}
{}
{\bfseries}
{.}
{ }
{Problem\thmnumber{ #2}\thmname{ #1}\thmnote{ (#3)} }
\theoremstyle{Problem}
\newtheorem{prob}{}
\newtheoremstyle{Solution}
{}
{}
{}
{}
{\bfseries}
{:}
{ }
{\thmname{#1}}
\makeatletter
\makeatother
\theoremstyle{Solution}
\newtheorem*{sol}{Solution}
% \usepackage{graphicx}


\begin{document}
	\thispagestyle{FirstPageStyle}
	
	\begin{prob}
		
	\end{prob}
	
	\begin{sol} 
	设$p$表示鱼大,q表示鱼刺大,r表示鱼肉少。则可以形式化为:\\
	1.p $\rightarrow$ q\\
	2.q $\rightarrow$ r\\
	3.r $\rightarrow$ $\neg$p\\
	4.p $\rightarrow$ $\neg$p\\
	步骤中非有效的点:
	\begin{enumerate}
		\item 第一句鱼的大小与于此大小无直接因果关系,仅能说同种鱼在同种环境条件下可能鱼越大,鱼刺越大。可以找到极端例子是的第一句前件真,后件假。
		\item 步骤2、3同样不具有有效性。能找到两种鱼,A的鱼刺\textgreater B的鱼刺,但A的肉\textgreater B的肉;也有肉少但体积大的鱼。
	\end{enumerate}
	故每一步均非有效。
	\end{sol}
	
	\begin{prob}
		
	\end{prob}
	
	\begin{sol} 
		{\bf 先证除了长度为2、3、6的合式公式都存在} $\forall$的单个命题符号长度为1,一元联结词$\neg$使得公式长度增加3(含括号)。设$A_1$、$A_2$为长度是1的命题符号,则$A_1,(A_1 \cap A_2),((A_1 \cap A_2) \cap A_2)$分别为长度是1,4,9的合式公式。结合取反长度就+3,可构造出长度为
		\begin{equation}
			\left\{
			\begin{aligned}
					length&=1,4,7...\\
					length&=5,8,11...\\
					length&=9,12,15...
			\end{aligned}
			\right.
			\end{equation}的合式公式。\\
		显然,除了2、3、6以外的长度的公式均已被包含,所以存在。
		
		{\bf 再证长度为2、3、6的合式公式都不存在} 设性质F(x)表示x这个合式公式长度并非2、3、6。假设$\alpha$,$\beta$满足F($\alpha$),F($\beta$),$\neg\alpha$,$(\alpha \star \beta)$($\star$表示二元联结词)长度显然不为2、3。假设二者中有长度为6的,可直接推出$\alpha$长度为3或者$\alpha$与$\beta$长度之和为3,这均不成立,所以假设不成立。所以长度为2、3、6的合式公式必不存在。
	\end{sol}
	
	\begin{prob}
		
	\end{prob}
	
	\begin{sol}
		\begin{enumerate}
			\item {\bf 奠基} 当$c = 0$时,无二元联结词,$\alpha$中命题符号只可能出现1次数(0次的话不是wff),形如A或者$\neg$A,所以$s = 1$,所以$s = c + 1$成立。
			\item {\bf 归纳假设} 假设当$c = k$的时候,$s = c + 1 = k + 1$。
			\item {\bf 归纳} 当$c = k + 1$的时候,表示在$ c = k$ 的基础上增加了1个二元联结词。设增加前的合式公式为$\alpha$,增加后为$\alpha^\prime$。不妨设增加的二元联结词为$\cap$,则变化为 $\alpha^\prime = (\alpha \cap A)$,其中A为命题符号。所以$s = (k+1)+1 = k + 2 = c + 1$。		
		\end{enumerate}
		所以$s = c + 1$得证。
	\end{sol}

	\begin{prob}
		
	\end{prob}
	
	\begin{sol}
		\begin{enumerate}
			\item {\bf 先证S中的$\forall$一个合式公式$\alpha$均在$S^\prime$中,即可以由$S^\prime$构造出$\alpha$:} 由S在5种运算下的封闭性可知:$\alpha$可以由S中的各个公式构造出来,$S^\prime$只需要依次构造即可。具体来说,若S由形如$A \star B$($\star$ 表示二元联结词),则可以讲A与B的构造序列进行拼接,并添加新元素$A \star B$,得到新的构造序列,依此可以构造出$\alpha$。而取否这样的变化亦是如此。
			\item {\bf 再证$S^\prime$中的所有构造序列产生的结论$\alpha$,均在S中:} 采用反证法:假设$S^\prime$中可构造出合式公式$\beta$,且$\beta$不在S中,而$S^\prime$构造又满足定义1.4的条件,即合法。那么这与S本身定义关于五种公式构造封闭相矛盾。所以$S^\prime$构造出的$\alpha$必在S中。
		\end{enumerate}
		综上:得证。
	\end{sol}


\end{document}